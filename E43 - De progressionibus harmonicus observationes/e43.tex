\documentclass[12pt, letterpaper]{article}

%\usepackage[letterpaper,left=0.7in,right=0.7in,top=\dimexpr15mm+1.5\baselineskip,bottom=0.7in]{geometry}
\usepackage{amsmath}
\usepackage{amsthm}
%\usepackage{pxfonts}
\usepackage[]{newpxtext}
\usepackage[varg,vvarbb]{newpxmath}
\usepackage{inconsolata}
\usepackage[T1]{fontenc}
\usepackage{float}
\usepackage{mathtools}
\usepackage{mathrsfs}
\usepackage{anyfontsize}

\usepackage{enumitem}
\usepackage{graphicx}
\usepackage{multicol}
\usepackage[table]{xcolor}

\usepackage{circuitikz}
\usepackage{pgfplots}
\pgfplotsset{compat=1.18}
\usepackage{wrapfig}
\usepackage{setspace}
\usepackage[export]{adjustbox}
\usepackage{tikzsymbols}

\usepackage[bottom]{footmisc}
\usepackage{hyperref}

\linespread{1.2}

\begin{document}
\title{Observations on harmonic progressions\footnote{Originally published as \emph{De progressionibus harmonicis observationes.} Commentarii academiae scientiarum Petropolitanae, Volume 7, pp. 150--161. Published in \emph{Opera Omnia}, Series 1, Volume 14, pp. 87--100.}}
\author{Leonhard Euler\footnote{Translation by Walter Jacob, \href{jacobw0@rowan.edu}{\nolinkurl{jacobw0@rowan.edu}}, Department of Mathematics, Rowan University.}}
\date{}
\maketitle

\subsubsection*{Translation notes.}
I have generally tried to maintain the tone of the original, while
translating the article into an idiom that can be understood by contemporary
English-speaking mathematicians. To this end, the term \emph{series} has been
translated as ``sequence'' when appropriate, and the \emph{etc.} commonly used
by Euler to denote continuation has been replaced by the contemporary ``\ldots''.
I have also added emphasis to new definitions, which is common in contemporary
textbooks and articles.
Otherwise, Euler's notation has been kept, with the exception that some
formulas are set in display mode to increase their readability. Any mathematical
errors are likely to be mine.

\subsubsection*{\S. 1.}
By the name of \emph{harmonic progressions} are understood all sequences of
fractions in which the numerators are equal among themselves and the
denominators form a true arithmetic progression.
Thus the general form is given by
$\frac{c}{a}$, $\frac{c}{a+b}$, $\frac{c}{a+2b}$, $\frac{c}{a+3b}$, \ldots.
Indeed, each three consecutive terms,
$\frac{c}{a+b}$, $\frac{c}{a+2b}$, $\frac{c}{a+3b}$,
have the property that the differences of the outer terms and the middle one
are proportional to the to the outer terms themselves. Namely, we have
\[
    \frac{c}{a+b} - \frac{c}{a+2b} : \frac{c}{a+2b} - \frac{c}{a+3b} = \frac{c}{a+b} : \frac{c}{a+3b}.
\]
Further, since this is the property of \emph{harmonic proportion}, these
sequences of fractions are called harmonic progressions.
They may also be called \emph{reciprocals of the first order}, since one is
subtracted from the index $n$ in the general term $\frac{c}{a+(n-1)b}$.

\subsubsection*{\S. 2.}
Although in these sequences the terms continually decrease, nevertheless the
sum of the terms of such a sequence continued to infinity is always infinite.
To demonstrate this, there is no need for a method of summing these series;
but the truth will easily be made clear from the following principle.
A series which, when continued to infinity, has a finite sum, even if it is
continued twice as long, will accept no increase, and that which is added
after infinity will actually be infinitely small.
For if this were not the case, the sum of the series, even if continued to
infinity, would not be determined and therefore not finite.
From this it follows that if that which arises from continuation beyond the
infinite term is of finite magnitude, the sum of the series must necessarily
be infinite.
From this principle, we can judge whether the sum of any given series is
infinite or finite.\footnote{
    This passage is more than a little cumbersome, thanks mostly to a lack
    of the formal idea of limits, but  Euler is giving a
    generalization of the standard undergraduate proof of the divergence
    of the harmonic series; namely, that you can always group some finite
    number of successive terms together to get a sum greater than or equal
    to some finite magnitude. In proofs for the divergence of the harmonic
    series, it is customary to show that such groupings result in sums
    greater than or equal to $\frac{1}{2}$. Here, the only difference is
    the $a$ in the denominator. To get rid of it, Euler will start at
    the "infinite" term, so that this $a$ is completely dominated and
    can be disregarded, leading to a convenient cancellation, as we will
    see in the next subsection. In contemporary usage, we would simply
    take "sufficiently large values" to neutralize this $a$.
}

\subsubsection*{\S. 3.}
Indeed, let the series $\frac{c}{a}$, $\frac{c}{a+b}$, $\frac{c}{a+2b}$, \ldots
be continued to infinity, and consider the term $\frac{c}{a+(i-1)b}$,
where $i$ denotes an infinite number, which is the index of this
term.\footnote{
    Euler starts at the "infinite" term, but really he just means that for
    sufficiently large indices $i$, the number $a$ will be dominated in such
    a way that it can be disregarded. He will use this in a few sentences to
    gather bounds on the sum of each consecutive $(n-1)i$ terms of the
    sequence.
}
Now let this series be further continued from the term $\frac{c}{a+ib}$ to
the term $\frac{c}{a+(ni-1)b}$ whose exponent is $ni$.
The number of these additionally appended terms is therefore $(n-1)i$;
their sum, however, will be less than $\frac{(n-1)ic}{a+ib}$ but greater
than $\frac{(n-1)ic}{a+(ni-1)b}$.
But since $i$ is infinitely large, $a$ will vanish in both denominators.
Hence the sum will be greater than $\frac{(n-1)c}{nb}$ but less than
$\frac{(n-1)c}{b}$.
From this it is clear that this sum is finite, and consequently the sum of
the given series $\frac{c}{a}$, $\frac{c}{a+b}$, \ldots, continued to infinity,
is infinitely large.

\subsubsection*{\S. 4.}
Further, tighter limits on the sum of the terms of the seqeunce from $i$ to
$ni$ are derived by the properties of harmonic proportions\footnote{
    Of $(n-1)i$ consecutive terms, as in subsection 1 where he examines three
    such consecutive terms.
}.
Namely, every harmonic proportion is such that the middle term is less than
one-third of the sum of all of the terms.
For this reason, the middle term between $\frac{c}{a+ib}$ and
$\frac{c}{a+(ni-1)b}$, which is $\frac{c}{a+\frac{ni+i-1}{2}b}$, multiplied by
the number of terms $(n-1)i$, or $\frac{(n-1)ic}{a+\frac{ni+i-1}{2}b}$,
will be less than the sum of the terms.
Therefore, the sum of the terms will be greater than $\frac{2(n-1)c}{(n+1)b}$
because $i$ is infinite.
Furthermore, the arithmetic mean between the extreme terms is greater than
one-third of the sum of the terms.
From this, it follows that in a harmonic series the sum of the terms will be
less than $(n-1)i$ times the arithmetic mean of the extreme terms multiplied,
which is $\frac{(2a+(ni+i-1)b)c}{2(a+ib)(a+(ni-1)b)}$ or $\frac{(n+1)c}{2nib}$.
Therefore, the sum will be less than $\frac{(n^2 - 1)c}{2nb}$, so that these
two limits are $\frac{2(n-1)c}{(n+1)b}$ and $\frac{(n^2 - 1)c}{2nb}$, and
thus the sum is approximately $\frac{(n-1)c}{b\sqrt{n}}$, which is the mean
between the limits.



\end{document}