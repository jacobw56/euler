\documentclass[12pt, letterpaper]{article}

%\usepackage[letterpaper,left=0.7in,right=0.7in,top=\dimexpr15mm+1.5\baselineskip,bottom=0.7in]{geometry}
\usepackage{amsmath}
\usepackage{amsthm}
%\usepackage{pxfonts}
\usepackage[]{newpxtext}
\usepackage[varg,vvarbb]{newpxmath}
\usepackage{inconsolata}
\usepackage[T1]{fontenc}
\usepackage[utf8]{inputenc}
\usepackage{float}
\usepackage{mathtools}
\usepackage{mathrsfs}
\usepackage{anyfontsize}

\usepackage{enumitem}
\usepackage{graphicx}
\usepackage{multicol}
\usepackage[table]{xcolor}

\usepackage{circuitikz}
\usepackage{pgfplots}
\pgfplotsset{compat=1.18}
\usepackage{wrapfig}
\usepackage{setspace}
\usepackage[export]{adjustbox}
\usepackage{tikzsymbols}

\usepackage[bottom]{footmisc}
\usepackage{hyperref}

\linespread{1.2}

\begin{document}
\title{De progressionibus harmonicis observationes.}
\author{Auctore \\ \emph{Leonh. Eulero.}\footnote{
        Copied by Walter Jacob.
        Every attempt has been made to create an accurate copy of the original
        paper, with a few exceptions. I have made the switch from consonantal
        \emph{u} to \emph{v}, where it is common (\emph{e.g.}, \emph{sive}
        instead of \emph{siue}) and I only use contemporary orthographical
        symbols (so no long \emph{s}). I also fixed several punctuation or
        capitalization errors; \emph{e.g.}, line 11 of subsection 3 in the
        original has a comma followed by, "\emph{Ex quo \dots}," which I
        have changed to a period. Otherwise, any remaining errors are mine.
    }}
\date{}
\maketitle

\subsubsection*{\S. 1.}
Progressionum harmonicarum nomine intelliguntur
omnes series fractionum, quarum numeratores sunt
aequales inter se, denominatores vero progressionem
arithmeticam constituunt. Huiusmodi ergo
forma generalis est
$\frac{c}{a}$, $\frac{c}{a+b}$, $\frac{c}{a+2b}$, $\frac{c}{a+3b}$, etc.
Quique enim tres termini contigui ut
$\frac{c}{a+b}$, $\frac{c}{a+2b}$, $\frac{c}{a+3b}$, hanc
habent proprietatem, ut differentiae extremorum a medio
sint ipsis extremis proportionales. Scilicet est
$\frac{c}{a+b} - \frac{c}{a+2b} : \frac{c}{a+2b} - \frac{c}{a+3b} = \frac{c}{a+b} : \frac{c}{a+3b}.$
Cum autem haec sit
proprietas proportionis harmonicae; vocatae sunt istiusmodi
fractionum series progressiones harmonicae. Vocari
etiam possent reciprocae primi ordinis, quia in termino
generali $\frac{c}{a+(n-1)b}$ index $n$ unicam eamque negativam
habet dimensionem.

\subsubsection*{\S. 2.}
Quanquam in his seriebus termini perpetuo decrescunt; tamen summa huiusmodi
serei in infinitum continuatae semper est infinita.
Ad hoc demonstrandum non opus est methodo hasce series summandi; sed veritas
facile ex sequente principio elucebit.
Series quae in infinitum continuata summam habet finitam, etiamsi se duplo
longius continuetur nullum accipiet augmentum, sed id quod post infinitum
adiicitur cogitatione, re vera erit infinite parvum.
Nisi enim hoc ita se haberet, summa seriei etsi in infinitum continuatae
non esset determinata et propterea non finita.
Ex quo consequitur, si id, quod ex continuatione ultra terminum infinitesimum
oritur, sit finitae magnitudinis, summam seriei necessario infinitam esse
debere.
Ex hoc principio iudicare poterimus, utrum seriei cuiusque propositae summa
sit infinita an finita.

\subsubsection*{\S. 3.}
Sit itaque series $\frac{c}{a}$, $\frac{c}{a+b}$, $\frac{c}{a+2b}$, etc. in
infinitum continuata, terminusque infinitesimus $\frac{c}{a+(i-1)b}$,
denotante $i$ numerum infinitum, qui sit index huius termini.
Iam haec series ulterius continuetur a termino $\frac{c}{a+ib}$
usque as terminum $\frac{c}{a+(ni-1)b}$ cuius exponens est
$ni$. Horum terminorum igitur insuper adiectorum numerus
est $(n-1)i$; summa eorum vero minor erit quam $\frac{(n-1)ic}{a+ib}$
maior vero quam $\frac{(n-1)ic}{a+(ni-1)b}$. Sed quia $i$
est infinite magnum, evanescet $a$ in utroque denominatore.
Quare summa maior erit quam $\frac{(n-1)c}{nb}$ at minor
quam $\frac{(n-1)c}{b}$. Ex quo perspicitur hanc summam esse
finitam, atque consequenter seriei propositae $\frac{c}{a}$, $\frac{c}{a+b}$,
etc. in infinitum continuatae summam infinite magnam.

\subsubsection*{\S. 4.}
Huius autem summae terminorum ab $i$ ad
$ni$ limites propiores ex sequentibus proportionis harmonicae
proprietatibus eliciuntur. Scilicet omnis proportio
harmonica its est comparata, ut terminus medius minor
sit quam pars tertia summae terminorum omnium. Hanc
ob rem terminus medius inter $\frac{c}{a+ib}$ et $\frac{c}{a+(ni-1)b}$, qui est
$\frac{c}{a+\frac{ni+i-1}{2}b}$, ductus in terminorum numerum $(n-1)i$,
seu $\frac{(n-1)ic}{a+\frac{ni+i-1}{2}b}$ minor erit quam summa terminorum.
Sive terminorum summa hinc maior erit quam
$\frac{2(n-1)c}{(n+1)b}$ ob $i$ inifinitum. Praeterea medium arithmeticum
inter terminos extremos maius est parte tertia
summae terminorum. Ex hoc sequitur fore etiam in
serie harmonica terminorum summam minorem quam
$(n-1)i$ in medium arithmeticum terminorum extremorum,
quod est $\frac{(2a+(ni+i-1)b)c}{2(a+ib)(a+(ni-1)b)}$ seu $\frac{(n+1)c}{2nib}$,
ductum. Quare summa erit minor quam $\frac{(n^2 - 1)c}{2nb}$, ita ut
hi duo limites sint $\frac{2(n-1)c}{(n+1)b}$ et $\frac{(n^2 - 1)c}{2nb}$, adeoque summa
proxime $= \frac{(n-1)c}{b\sqrt{n}}$ quod est medium proportionale inter
limites.


\end{document}